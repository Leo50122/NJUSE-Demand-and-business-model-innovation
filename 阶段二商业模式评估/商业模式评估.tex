\documentclass[a4paper,12pt]{article}

% 导言区
\usepackage{titlesec}
\usepackage{lipsum} % 示例用,可以删除
\usepackage{geometry}
\usepackage{setspace}
\usepackage{amsmath} % 用于数学公式
\usepackage{graphicx} % 用于插入图片
\usepackage{float}
\usepackage{lipsum} % 用于生成虚拟文本
\usepackage{ctex} % 导入 ctex 包以支持中文
\usepackage{titlesec} % 导入 titlesec 包以定制标题样式
\usepackage{fontspec} % 用于设置中文字体
\usepackage{amsfonts}
\usepackage{amsmath} % 提供 \text 和 \tanh 命令
\usepackage{bm}      % 提供 \bm 命令用于粗体
% 目录设置
\usepackage[nottoc,notlot,notlof]{tocbibind}
\usepackage{enumitem}
% 页面设置
\geometry{margin=1in}

% 标题设置
\titleformat{\section}{\normalfont\Large\bfseries}{\thesection}{1em}{}
\titleformat{\subsection}{\normalfont\large\bfseries}{\thesubsection}{1em}{}
\titleformat{\subsubsection}{\normalfont\normalsize\bfseries}{\thesubsubsection}{1em}{}

% 行间距设置
\onehalfspacing

% 文档信息
\title{商业模式评估}
\author{需求不寄小分队}
\date{\today}

\begin{document}

    \maketitle

% 添加目录
    \tableofcontents


    \subsection{团队成员}
    211250124 程智镝

    211250122 刘辉

    211250159 陈凌

    211250158 李忠信
    \subsection{度量数值}

    \section{文档简介}
    

        \section{商业模式环境}
    \subsection{市场影响力}
    \subsubsection{市场影响力}
    \textbf{子问题1:影响客户环境的关键问题有哪些?}
    
    如何体现:哪些关键问题正在影响会议软件的市场?
    
    回答:在会议软件中,延迟与画⾯清晰度是⼀⼤关键。优质的云服务商能提供强⼤的云服务平台与快速访问的渠道,使得会议视频的速度与质量得到保证。会议的各项功能能否满⾜客户需求是我们必须考虑的⼀⼤问题。企业用户对于会议软件的定制化需求不断增加,因此软件需要提供更多可配置和可扩展的选项,以满足不同企业的需求。会议软件市场的竞争激烈,价格和付费模式成为用户选择的关键因素。提供灵活的定价计划,并确保性价比是维持市场份额的重要手段。
    
    \textbf{子问题2:哪些改变正在发生?}

    如何体现:当今的会议软件正在经历怎样的变化?

    回答:在后疫情时代,远程会议软件依然有着蓬勃的发展力。例如新闻1中提到,腾讯会议在上线2个⽉后,⽇活跃度已经突破1000万,⽽即便是现在公司以及学校已经复⼯和复学之后,腾讯会议依然处于强势的扩张期。在疫情扩散与慢慢趋于稳定后,崛起的许多会议软件也⾯临危机,客户已经⽆法满⾜旧有的视讯功能,会议软件需要注重提供支持远程协作和沟通的功能,而不再仅仅是单独的视讯与⾯对⾯交流,新闻2就提及,疫情给远程会议⼀个有拥有⼤量⽤户的机会,但是如何在疫情之后仍然留住这些客户,还需要根据⽤户反馈进⾏实时的调整与更新。

    \textit{调研新闻 1——⾃上线2个⽉⽇活破千万后,腾讯会议继续扩⼤着⾃⼰的影响⼒。在上线245天后,成为中国最多⼈使⽤的视频会议产品,并多次登顶APPSTORE中国区免费榜。}
    
    \rightline{\textit{——腾讯⽹:云会议市场峰回路转,远程办公迎来“腾讯时刻”?}}

    \textit{调研新闻 2——⾏业需要根据⽤户反馈的问题进⼀步做出相关调整,⽐如利⽤⼿机和电脑等硬件来开视频会议,相对还有⼀些阻碍,相关的产品或服务会有⼀个逐渐演化、完善的过程。}
    
    \rightline{\textit{———中国经营报:远程视频会议⽕爆,前景价值⼏何?}}

    
    \textbf{子问题3:市场将走向何处?}

    如何体现:对于现今的视频会议软件,未来的发展趋势是什么?

    回答:会议软件的盈利模式来⾃于客户的消费与会员制度,会员制度中最需要确保的客户群是⼤公司与⼤企业的使⽤率,如果会议的形态不积极迎合客户群体,那么迅速的被取代是必然的结果。因此,会议的客户端与功能必须根据客户的反馈进⾏实时的优化与调整。凭借远程会议方便快捷的优点,对许多企业来说,使⽤线上会议进⾏开会是转型的趋势,这时,哪⼀个软件功能⻬备与体验更好成为企业选择的标准。未来的视讯会议软件必须包装更多的功能,不能只是单纯的追求通讯流畅与扩增会议⼈数,需要向更多元的⽅向去发展。如同新闻3所说,腾讯会议正因不断迭代的功能,才能不断吸引新的客户群体。

    \textit{调研新闻 3——腾讯会议在技术和⽤户体验上持续改进,通过最可靠的安全策略,快速迭代完善的产品功能,为更多企业的远程办公保驾护航。}

    \rightline{\textit{——腾讯⽹:⼀⽂告诉你,腾讯会议为何能成为政务企业的⾸选}}
    
    \subsubsection{市场分类}
    \textbf{子问题1:哪些是最重要的客户群体?}

    如何体现:在商业模式画布中的客户细分中,哪⼀部分客户是最重要的客户?

    回答:在客户群体⾥⾯,能提供⼤量资源与商业利益的必然是⾸先服务的对象。如果⼀整个企业都是我们的客户群体,那么能带来的利益是⾮常庞⼤的。从新闻4中得知,腾讯会议总是⾯向企业服务,为企业打造最优质的办公环境与线上的会议环境。腾讯会议也因为优秀的保密算法与强⼤的云服务平台,在新闻5⾥⾯能够吸引超过1亿⼈的⽤户群体,⽽且最⾼能⽀持2000⼈在线,同时发布了腾讯会议企业版来吸引企业用户使用。这也表明了⼤客户群体才是⾸要考虑的市场,企业级的客户能带来的流量与资⾦⾮常可观,能确保前⾯提到的版本迭代有足够的资⾦与资源顺利推⾏。

    \textit{调研新闻 4——腾讯会议已有超过1亿注册⽤户,企业微信⽇活跃账户数同⽐增⻓超过100\%。}

    \rightline{\textit{——中国新闻⽹:腾讯Q3财报:腾讯会议⽤户数破亿,云业务加速⽴标杆}}

    \textit{调研新闻 5——作为中国⽬前最多⼈使⽤的专业视频会议产品,腾讯会议已深度服务于政务、⾦融、教育、医疗等⾏业和中⼩企业在线办公,为企业级⽤户打造专属会议能⼒,最⾼⽀持2000⼈同时在线。腾讯发布腾讯会议企业版本,增强了⽹络研讨会,同声传译及连接企业原有会议室系统等功能。}

    \rightline{\textit{————中国青年⽹:Q3腾讯会议⽤户过亿 ⾦科及企业服务板块增⻓24\%}}
    
    \textbf{子问题2:哪个群体的增长潜力最大?}

    如何体现:在商业模式画布中的客户细分中,那⼀部分的客户是可积极发展商业化的⽬标群体?

    回答:最具商业潜⼒的群体⽆疑是⾼校师⽣。许多⾼校发现了线上授课的诸多好处,新闻7中就提到,数字化教学低成本且⾼效,可以进⾏跨区域的协作,是⼀个待开发的潜⼒客户群体。⽬前市⾯上⼤型的会议软件都没有特别推出教育版,可以专⻔对课堂进⾏适配,所以这是⼀个待开发的群体,良好的⾼校合作可以带来庞⼤的商业利益。
    当今时代,青少年的网络普及率极高,可以说,⽹络产品的客户群体年龄层还是⾯向年轻⼈族群。⻘少年更能接受⽹络带来的改变,⽽且更愿意在⽹络上消费、购买虚拟产品以及虚拟服务,不仅是教育层面的发展,在面向青少年娱乐性质的会议,也具有很大商业化潜力。

    \textit{随着IT技术的逐步完善,我国的信息化教育也进⼊第⼆阶段,其中最典型的配合就是⾼清视频会议系统软件应⽤远程教育了,它完美的契合数字化教学的应⽤,其中就包含在线流媒体教育、远程观摩评价、远程互动教育、远程会议与培训、跨区域协作等。}

    \rightline{\textit{——腾讯⽹:⾼清视频会议系统软件助⼒教育⾏业应⽤}}

    \textbf{子问题3:哪些边缘群体值得留意?}

    如何体现:在商业模式画布中的客户细分中,那⼀部分的客户是占比较小但值得发展的⽬标群体?

    回答:如果腾讯会议在国际市场扩张,国际用户可能是一个边缘群体,特别是那些在不同时区工作的组织和专业人士。
    如果能够提供特定行业或专业领域所需的功能和服务,例如医疗、法律等,那么这些领域的用户可能成为关注的群体。
    在一些新兴市场,尤其是亚洲和非洲等地区,在支持本地语言和文化的情况下,腾讯会议可能找到增长机会。
    \subsubsection{需求与诉求}
    \textbf{子问题1:客户需要什么?}

    如何体现:客户想通过会议软件解决什么问题?

    回答:⾼校师⽣可以⽤它来进⾏远程教学、企业可以进⾏保密的线上会议、跨国⼤企业可以进⾏不断联且低延迟的通讯会议。
    
    \textbf{子问题2:客户真正想要搞定什么?}

    如何体现:客户使⽤我们的产品时最基本的需求是什么?

    回答:客户使⽤远程会议软件的最根本⽬的就是快捷与⾼效。在新闻9中明确的点出,远程会议拥有着不可取代的优势,不经能减少⼈员出差的成本,同时也打破了地域限制,在不同国家、不同地域的两群⼈,也能随时随地、零延时的进⼊会议,远程会议若使⽤得当,可以⼤幅度的节省成本,并提⾼企业的⽣产效率。

    \textbf{子问题3:哪些需求在增加?哪些在减少?}

    如何体现:在⼦问题1中的需求,哪⼀个需求时呈现增⻓的趋势?哪些需求其实对⽤户⽽⾔会逐渐变得不再重要?

    回答:商业中的跨国会议业务会议需求正在不断的增加。许多事情与许多业务牵扯的不仅仅是⼀个国家的企业,在产业链上⾯,可能包含着全球各地的供应商与服务商,这时候,跨国间的沟通就显得格外重要。想要更好、更顺畅的让跨国会议顺利进⾏,软件的跨国业务与选择的服务器供应商也必须到位,能⽀持⼤量的跨国数据传输并保持良好的低延时与⾼画质的会议品质。当然,多语⾔的⽀持也是必须的,因此需要⼤量的翻译⼈员与⼤量的⼝语服务员能够进⾏客户沟通,⼈⼒以及资源的成本都是亟待解决的问题。
    
    而对于减少的需求,⾃从复学之后,⾼校的需求正在急剧下滑。因为⼤部分的⾼校师⽣已经回到校园进⾏⾯对⾯的授课,教学会议变得不再重要,所以软件必须积极进⾏转型,将⾯向的群体从原本的校内师⽣转为像是mooc那样的教学⽅法,⽼师可以督促学⽣进⾏学习,⽽且可以减少⽼师很多的⼈⼒批改时间,我们必须将产品在教师⽤户⼼中的定位从不得不⽤该产品上课,转为想使⽤该产品上课,必须将⽹络上课对⽼师的好处体现出来,这样才能留住教育群体。
    
    \subsubsection{切换成本}
    \textbf{子问题1:哪些东西将客户捆绑在一家供应商和它的服务上?}

    如何体现:有哪些服务是腾讯会议可以提供给客户⽽其他公司提供不了的?

    回答:腾讯会议作为腾讯生态系统的一部分,可能更容易与其他腾讯产品和服务进行集成,例如腾讯云、腾讯邮箱、企业微信等,为用户提供更为一体化的工作体验。
    腾讯会议提供便捷的云端录制和存储功能,用户可以方便地保存和分享会议录音和录像,这有助于追溯会议内容和提高信息传递效率。
    同时,腾讯会议具备高质量的虚拟背景和美颜功能,为用户提供更为个性化和美观的会议体验,这些功能在一些用户对外观和形象有特殊要求的场景中可能更受欢迎。腾讯会议在企业版中提供一系列个性化定制选项,以及适用于企业级管理的功能,如会议权限控制、用户管理等,以满足企业客户的特定需求。

    \textbf{子问题2:哪些切换成本阻止客户转投竞争对手?}

    如何体现:⽤户如果不使⽤我们的产品,⽽去使⽤其他的社交软件或者视频软件 ,他会失去什么?

    回答:腾讯会议专注于提供高质量在线会议和协作工具的平台,具有专业的会议功能,例如屏幕共享、实时聊天、远程控制等。使用其他社交软件或视频软件可能无法获得同样的专业会议体验。腾讯会议通常提供企业级的安全性和隐私保护措施,适用于敏感业务和机密信息的在线沟通。其他社交软件或视频软件可能没有同样级别的安全保障。如果用户在腾讯生态系统中使用其他服务,例如腾讯云、腾讯文档等,腾讯会议可能更好地集成这些服务,提供更为无缝的协作体验。


    \textbf{子问题3:客户容易找到并采购类似的服务吗?}

    如何体现:现在市场上有和腾讯会议提供相似的服务的软件吗?

    回答:目前中国市场中视频软件数量较少,但是在国外有很多类似软件例如在全球范围内广泛使用,尤其是在远程办公和在线教育方面的Zoom;微软提供的团队协作和沟通工具,集成在Microsoft 365套件中的Microsoft Teams;以及老牌即时通讯软件Skype。

    \textbf{子问题4:品牌有多重要?}

    如何体现:竞争市场上有巨头式的知名软件吗?腾讯会议打造了怎样的品牌?

    回答:⽬前使⽤的视讯会议软件最⼤的巨头是Zoom,腾讯会议致⼒于弥补Zoom的缺点并改善,在此基础上加以中国式创新,以期创造⼀个能够适配⼤量特别面向国内⽤户群体的兼容性的软件。

    \subsubsection{收入影响力}
    \textbf{子问题1:客户真正愿意花钱买的是什么?}

    如何体现:腾讯会议提供的服务中客户愿意为什么服务付费?

    回答:⾼品质的使⽤体验、优秀的加密算法、云储存空间收费、更加个性化的设置、更多的参会⼈数。除开收费内容之外,腾讯会议仍然支持⼤批的免费⽤户使⽤。
    
    \textbf{子问题2:利润中最大的一块从哪里获得?}

    如何体现:腾讯会议在收⼊来源中那⼀部分的获利是最⼤的?

    回答:腾讯会议提供了不同的订阅服务,企业用户可以选择适合其需求的付费版本,享受更多高级功能、更大规模的参会人数限制以及其他专业服务。腾讯会议提供一些高级功能和附加服务,需要用户付费使用,例如虚拟背景、云端录制、实时翻译等。腾讯会议可能针对大型企业提供一些定制的企业级解决方案,包括更深度的集成、安全性定制等服务。
    
    \textbf{子问题3:客户能够轻易找到并购买更便宜的产品和服务吗?}

    如何体现:有没有同类型的竞争软件会⽐腾讯会议收费更低?

    回答:会员订阅服务费用的收取已经是⽐较常⻅的收费⽅式,唯⼀能⽐这个还便宜的就是免费的软件。但免费软件往往会夹杂许多⼴告,甚⾄窃取个⼈信息,服务质量也大不尽如人意。最终是没法和腾讯会议相⽐的。
    
    \subsection{行业影响力}
    \subsubsection{(现有的)竞争对手}
    \textbf{子问题1:谁是我们的竞争对手?}

    如何体现:现在已有的市场环境中,谁会与我们的业务产⽣冲突,他们为什么会被我们视为直接的竞争对⼿?

    回答:ZOOM是我们的竞争对⼿,是全球视频会议服务⾏业的领头⽺,在视频会议领域已经有了⼀定的⽤户群体,有较为稳定的收⼊来源,同时客户已经对他产⽣了⼀定的依赖。它成名已久,拥有众多的企业级⽤户,是我们的直接竞争对⼿。
    

    \textbf{子问题2:哪些是领域内的主流玩家?}

    如何体现:识别视频会议领域内的主流软件

    回答:⽬前在领域内,国内的教学会议会使⽤⾬课堂,国外企业级会议是Zoom等软件。
    \textbf{子问题3:他们的竞争优势或劣势是什么?}

    如何体现:主要竞争对⼿的优势和劣势是什么?

    回答:在国外的企业级会议中市场普及率以及资源供给最为完善的就是Zoom。Zoom作为⼀个企业级视频会议服务提供商,也已经有了⼀定的规模,有稳定的⽤户群体。腾讯会议进⼊国际市场之后较难短时间内抢占⽤户群体。Zoom软件在使⽤过程中视频流的安全性已经出现多次的问题,在视频会议软件特别是企业级的会议中出现⽹络安全问题后果通常⽐较严重。同时国内的教育会议的供应商还有⾬课堂,他们的优势是瞄准了国内的教育市场,有精确的定位,有专⻔针对课堂的功能,能够供软件使⽤⼈员使⽤。但是据⼤部分学⽣反馈,视频会议时不时会出现资源供应不⾜以及崩溃的问题,严重影响了⽤户的使⽤体验。

    \textbf{子问题4:描述他们的主要产品和服务}

    如何体现:描述他们的主要产品和服务

    回答:Zoom Meetings是Zoom的核心产品,提供高质量的视频会议和音频通话服务。用户可以通过Zoom Meetings进行远程协作、在线会议和虚拟沟通。Zoom Rooms 是Zoom专为大型会议室和协作空间设计的服务。它提供了硬件和软件的整合,使得在大屏幕上进行视频会议更为便捷,适用于企业内部和团队协作。雨课堂提供各种在线课程,涵盖了多个领域,包括职业培训、技能提升、兴趣爱好等。用户可以通过雨课堂平台报名参加这些课程,学习内容包括视频教学、文档分享、在线测试等。部分雨课堂的课程可能是付费的,用户需要支付相应的费用以参与这些课程。这些付费课程可能提供更深入、更专业的教学内容,适用于有特定学习需求的用户。

    \textbf{子问题5:他们聚焦哪些客户群体?}

    如何体现:上述软件的客户细分?

    回答:Zoom聚焦于企业会议,向个⼈⽤户提供收费服务。⾬课堂聚焦于教育会议,旨在向师⽣提供云上课堂的环境。
    
    \textbf{子问题6:他们的成本结构如何?}

    回答:Zoom提供了免费版本,但也提供了不同级别的订阅服务,用于满足更高级别和大规模使用的需求。企业和个人用户可以选择不同的订阅计划,根据其需求支付相应的费用。 Zoom的企业用户可以选择购买企业级订阅,这些订阅通常包括更多的高级功能、更大的会议容量、更长的会议持续时间等。雨课堂提供各种在线课程,包括职业培训、技能提升、兴趣爱好等。一部分课程可能是付费的,用户需要支付相关费用以参与这些课程,提供付费会员服务,让用户享受一些特殊的会员权益,例如更多的免费课程、专属活动等。
    
    \textbf{子问题7:他们对于我们的客户群体、收益来源和利润有多大影响?}

    如何体现:对我们的客户细分、收益来源、利润有何影响

    回答:由于Zoom在国内停⽌服务的原因,Zoom在国内的⽤户群体会逐渐脱离Zoom,开始使⽤国内的视频会议平台进⾏会议,这是我们的⼀部分客户来源。同时Zoom会议在近期频繁爆出的安全问题,也使Zoom的⼝碑出现下滑,使⽤Zoom的企业不会选择Zoom来进⾏企业⾼机密的交流。⾬课堂对于教育领域的精细化⽀持可能会导致我们的客户群体的下滑,但是其产品的崩溃问题的存在⼀直是⼀个⾮常严重的使⽤体验上的不⾜。同时腾讯会议不仅仅只是对教育区块进⾏精细化的服务,所以我们的客户群体不会与他们直接产⽣冲突。
    
    \subsubsection{新进入者(挑战者)}
    目前视频会议软件的新进入者并不多,或者是没有⼀定的市场规模足以抢占市场。
    \subsubsection{替代产品和服务}
    \textbf{子问题1:哪些产品和服务能够替代我们的产品和服务?}

    回答:Zoom以及⾬课堂

    \textbf{子问题2:客户要切换到这些替代品有多容易?}

    回答:客户在使⽤腾讯会议的过程中基本上没有产⽣依赖平台的数据,现阶段的视频会议软件基本上都是对视频会议进⾏服务,没有数据存储的功能(腾讯会议正在增加这项功能)这就导致了客户是可以直接进⾏使⽤平台的切换,⽬前可以预知的切换平台的唯一阻⼒来⾃于⽤户尚未⽤完的会员服务订阅期限。

    \textbf{子问题3:这些替代产品起源于何种商业模式传统?}

    回答:腾讯会议起源于传统的视频通信和在线协作的商业模式,融合了视频通信、在线协作和SaaS模型等多个传统商业模式的元素,以满足现代远程办公和协作的需求。随着在线工作趋势的增强,这样的在线会议工具在商业领域得到了广泛应用。
    
    \subsubsection{供应商和价值链上的其他厂商}
    \textbf{子问题1:谁是你的行业价值链中的关键玩家?}

    回答:
    
    1. 腾讯公司: 作为腾讯会议的母公司,腾讯在整个价值链中扮演着核心角色。它负责产品的研发、运营、市场推广以及整合腾讯的生态系统资源,包括腾讯云、企业微信等。
    
    2. 腾讯云: 腾讯云是提供云计算基础设施和服务的关键组成部分。腾讯会议的视频会议服务可能倚赖腾讯云提供的稳定、高效的云计算资源。

    3. 合作伙伴: 腾讯会议可能与各种合作伙伴合作,包括硬件制造商、行业解决方案提供商、培训机构等,以丰富其产品生态系统,提供更全面的解决方案。

    \textbf{子问题2:你的商业模式在多大程度上依赖其他这些玩家?}

    回答:腾讯会议的商业模式在很大程度上依赖其他玩家,它的商业模式是一个生态系统,多方协同合作构成了其成功的基础。腾讯公司内部的资源,尤其是腾讯云提供的云服务,以及与合作伙伴和企业用户的协作,都对腾讯会议的发展和用户体验起到了关键作用。这种多方合作的商业模式有助于构建更全面、可持续的产品生态系统。

    \textbf{子问题3:有边缘玩家在涌现吗?}

    回答:边缘玩家包括与腾讯会议有间接联系或在其生态系统外提供相关服务的企业或组织。例如提供视频摄像头、音频设备、显示器等的硬件公司、开发与腾讯会议集成的第三方应用程序的开发者、向腾讯会议提供快速的网络服务的网络服务提供商。

    \textbf{子问题4:哪个的利润最高?}

    回答:如果硬件制造商能够提供高质量、创新且与腾讯会议完全兼容的硬件设备,例如高清摄像头、先进的音频设备等,他们可能会在销售中获得较高的利润。这尤其在企业级市场中可能更为显著。
    
    \subsubsection{利益相关者}
    \textbf{子问题1:哪些利益相关者会影响你的商业模式?}
    
    回答:在客户使⽤我们的软件的时候如果使⽤额外的录屏软件以及其他的云平台进⾏存储可能会导致盈利降低。由于海内外版权的分割也容易造成⼀定的影响,可能导致跨国的⽤户难以进⾏娱乐⽅⾯的共享,所以我们旨在通过让版权⽅直接⼊驻平台的⽅式进⾏产品的推⼴。

    \textbf{子问题2:股东的影响力如何?}

    回答:对于腾讯会议来说,其主要股东是腾讯公司,腾讯公司对腾讯会议的影响力是非常显著的。腾讯公司作为主要股东,会对腾讯会议的战略方向和发展规划产生直接的影响。腾讯公司的投资和资源分配决策会直接影响到腾讯会议的资金、人力资源等方面。腾讯公司作为中国科技巨头,在市场中具有强大的影响力。其对腾讯会议的支持和参与可能对产品的市场地位和竞争优势产生积极的影响。股东在公司治理中通常有一席之地,通过董事会或其他决策机构参与公司的决策过程。


    \textbf{子问题3:员工呢?政府呢?游说者呢?}

    回答:腾讯会议需要⼤量的优质服务团队的,在我们的售后服务团队中需要⼤量的售后服务⼈员和客户保持良好的沟通环境。
    腾讯会议需要密切关注并遵守政府的相关政策和法规,以确保其业务的合法性和可持续性。政府的政策变化和法规调整可能对腾讯会议的运营策略和业务模式产生重大影响。说客可能通过与政府官员、政策制定者等相关方沟通,试图影响政策制定、法规制定和监管措施,从而对腾讯会议的经营环境产生影响,因此腾讯会议需要通过积极参与公共政策讨论、与政府机构保持沟通,以及与利益相关方合作,来有效应对游说者对其业务环境的影响。
    
    \subsection{关键趋势}
    \subsubsection{技术趋势}
    \textbf{子问题1:你的市场内外的主要技术趋势有哪些}

    如何体现:当今视频会议软件的市场主要技术趋势⾛向有哪些?

    回答:视讯会议软件的技术组成是云服务+加密算法+⾼效的传输算法,与好的云服务商合作,将SaaS最⼤化,使⽤他们优秀的分布式计算系统来降低单⼀服务器的流量来避免流量过载,⽽优秀的加密算法能避免数据泄露,让更多的企业放⼼合作,不必担⼼公司机密外泄的问题,⽽⾼效的传输算法则能保证跨国会议的质量,打造零延时、⾼质量的会议体验。会议中最⼤的问题还有设备兼容的问题,因为现在基本上⼈⼿⼀台⼿机,⼿机开会和电脑开会必须互通是通讯软件必须解决的问题,毕竟两者基于不同的操作系统与不同的硬件架构,要能完全兼容是⼀⼤难题。

    \textit{调研新闻——想和客户开⼀个多⽅⼤型线上会议时,却发现两边的视频会议系统,因品牌不同不兼容,⿎捣了半天也没能连上。犹记得⽼板指着视频会议系统训IT⼩哥的画⾯......}

    \textit{\rightline{——科技每⽇推送:要开⼀个多⽅远程会议,有多难?}}

    \textbf{子问题2:哪些技术代表了重要的机会或者颠覆性的威胁}

    如何体现:当今视讯会议软件的技术有什么发展和阻碍?

    回答:现今云服务商的扩增与⼴泛使⽤能⼤⼤的促进会议软件的发展,更能打造零延时、⾼质量的视讯会议。⽹络平台的发展促进了数据的⼤范围流动,但同时太多数据在⽹络平台上传播,要是有更强的⿊客发现保密算法的漏洞,⼀旦遇到攻击,机器和设备都会崩溃,数据安全岌岌可危。

    \textbf{子问题3:哪些新兴技术是边缘客户正在逐步采用的}

    如何体现:当今视讯会议软件有什么可以添加的技术?

    回答:如果⾃然语⾔处理的技术能更加进步,可以进⾏会议内的同声传译,也就是⽤⾃然语⾔识别来让跨国会议中的交流变得更加顺畅,云计算加上⾃然语⾔处理并翻译可以增加⼤量的⽤户群,并且减少会议时对⽂本⽂件以及⼝语进⾏翻译的时间与⾦钱成本。

    \subsubsection{行业管理趋势}
    \textbf{子问题1:哪些管理趋势会影响你的市场}

    如何体现:当今视讯会议软件的管理会影响我们的平台发展?

    回答:视讯会议是虚拟空间,但是并不是法外之地,仍需要收到法律规范,所有开会的⾔辞与流传的数据不能有⾮法的操作,虽说我们对客户的保密性做的事⾮常到位的,基本不会泄露客户资料,但是如果说客户在视讯中讨论⼀些⾮法的交易或是在会议中公然播放不雅视频,我们也会通知相关部⻔进⾏法律制裁。尤其是当⾃⼰的⾔论与国家安全相抵触时,应⾃觉以国家安全为重。我国关于社交⽹络的现⾏⽴法层次较低,以部⻔规章或地⽅法规为主,缺少⾼位阶、统领性的安全⽴法。这样的管理有利于社交平台保留更有价值的观念,减少争论以及舆论的发展。
    
    \textbf{子问题2:哪些规则会影响你的商业模式}

    如何体现:我们的平台会受到那些规则的阻碍?

    回答:我们会强烈抵制盗版影⽚在会议中播放,⼀旦发现则⽴即终⽌该会议并警告播放会议的⽤户,我们会如同新闻⼀般,配合有关部⻔进⾏监管版权问题。⽽我们也重视客户的隐私权,绝对不会将客户的信息数据作为商品卖给其他利益团体,更使⽤最优秀的算法对客户信息进⾏加密,不然⿊客有机可乘。⽽在我们的会议严禁播放不雅影⽚或是在会议中进⾏⾮法交易,⼀旦被发现或是举报,将会通知有关部门进⾏处理。我们更会遵守所有国家的法规,在跨国会议中避免触及各国法律,并且遵循正规渠道传输跨国数据。

    \textit{调研新闻——“2018年10⽉21⽇消息,国家版权局直接对 20 家⼤型视频⽹站、 20 家⼤型⾳乐⽹站、 8 家⽹盘、 10 家⼤型⽂学⽹站开展了版权重点监管。”}

    \textit{\rightline{————亿邦动⼒讯}}

    \textbf{子问题3:哪些管理规定和税费会影响客户需求}

    如何体现:我们的平台会因为规则限制⽽影响客户需求吗?

    回答:会议软件主要根据他们的地区定制会费,毕竟各个国家的增值税与关税不同,根据国家定制符合客户购买⼒的价格,并不会定制过⾼的价码使得⽤户⽆法接受。⽽作为视讯平台,⽽且具有开放会议功能,我们必须确保我们播放的视频是合法且年龄分级妥当的,因此,若是有⼈在会议中播放⼀些不合乎当前国家规范的视频,我们会⽴即进⾏处理并对客户进⾏警告,我们最⼤化客户的⾃由,但是我们仍然不允许客户进⾏违法的操作,即使会影响该客户的需求。

    \subsubsection{社会和文化趋势}    
    \textbf{子问题1:文化或社会价值观上的哪些变化会影响你的商业模式}

    如何体现:平台的价值观、⽂化理念的哪些变化会影响产品发展?

    回答:随着消费者的社会意识的提升,对于平台的社会形象好坏以及宣传出去的价值理念都会进⾏考量。腾讯会议致⼒于打造⼀款拥有全⽅位功能的虚拟会议软件,在加强形象⽅⾯的拓展与⼝碑,建⽴起优秀的品牌形象。并且根据各个⽂化调整营销策略。我们坚决抵制不良或是影响观感的负⾯形象,希望能打造全⾯且优质的品牌形象,树⽴业界标杆。

    \textbf{子问题2:哪些趋势会影响购买者的行为}

    如何体现:什么样的世界变化会影响客户对视讯会议软件的使⽤?

    回答:软件受益于当今的⽹络发达以及疫情的爆发,可谓因祸得福,占尽了天时地利,像是腾讯就好好的把握这次机会,如同新闻17所提及,在短短245天就突破1亿⽤户。增长的如此之快得⼒于业务的快速扩张,在新闻18中就直接点明,腾讯会议在短短⼏天内扩容超过10万台云主机,投⼊超过百万核的计算资源,才能迅速崛起,成为超过100个国家和地区的会议平台,并带动了相关产业的发张,成功拓宽市场并且带动腾讯的相关产业链如腾讯云的业务上升。可以⻅得,世界的变化会⼤幅度的影响并促进了会议软件的使⽤率与服务⾯,⽽能不能抓住这种机会⼤量拓展业务并吸收新进的客户群体,是我们必须要注意的产品发展⽅向。

    \textit{调研新闻 17——QQ 达成⽤户数破亿花了四年,微信耗时⼀年多,⽽腾讯会议只⽤了 245 天,成为腾讯近年来⽤户增速最快的产品。}
    
    \textit{\rightline{——新浪科技:上线 245 天⽤户数破亿,腾讯制胜视频会议时代的⿊⻢产品出现了?}}

    \textit{调研新闻 18——8天时间⾥,腾讯会议总共扩容超过10万台云主机,投⼊超过百万核的计算资源,完成了全球超过100多个国家和地区的服务覆盖。此举创下了中国云计算史上前所未有的纪录,体现了科技产品的中国速度}
    
    \textit{\rightline{——新浪财经:致⼒于构建⽹络连接共同体,腾讯会议获得“世界互联⽹领先科技成果”奖}}
    
    \subsubsection{社会经济趋势}
    \textbf{子问题1:关键的人口统计学趋势有哪些}

    回答:中国的人口逐渐老龄化,即老年人口比例逐渐增加。这是由于长寿命和生育率下降等因素导致的。老龄化趋势对社会养老、医疗等方面提出了挑战。由于一孩政策的实施,曾经存在过性别比例失衡的问题,即男女比例不平衡。政府采取了措施来缓解这一问题,但在一些地区仍然存在性别比例失衡的情况。随着经济的发展,中国的教育水平不断提高。更多的人获得高等教育,这对国家的人力资源和创新能力具有积极的影响。传统的大家庭结构逐渐演变为核心家庭结构,即父母和子女的小家庭。这对家庭关系、社会福利等方面带来了新的挑战。中国有大量的农民工和流动人口,他们从农村流向城市寻找工作和机会。这种流动对城市和农村经济、社会服务等方面都带来了影响。

    \textbf{子问题2:你的市场中收入和财富的分布有哪些特征}

    回答:在中国,城乡之间存在显著的收入差距。城市通常拥有更多的就业机会、更高的薪资水平,而农村地区的居民通常面临着相对较低的收入水平。东西部差距: 中国东部沿海地区相对较富裕,而西部地区相对较贫困。这导致了地区之间的巨大经济差异,特别是在城市化和工业化方面。在城市中,高技能职业和高级管理人员通常享有较高的收入水平,而低技能工人和非技术性职业的收入相对较低。这导致了在城市内部的社会阶层差异。随着经济的发展,中国社会不断发展成为一个更加多层次的社会结构,不同社会阶层之间的收入和财富差距也在逐渐显现。

    \textbf{子问题3:描述你所处市场的消费特征}

    回答:随着人们收入水平的提高和生活水平的改善,中国消费者逐渐从基本生活需求转向更为多样化、品质化的消费。越来越多的人注重品牌、服务质量和个性化消费体验。中国是全球最大的数字市场之一,数字支付和电子商务得到了广泛应用。消费者越来越倾向于在线购物、移动支付和数字化服务,这反映在诸如支付宝、微信支付、京东等平台的快速发展。健康意识的提高使得健康食品、运动器材和生活方式产品的市场需求增加。这也反映在对有机食品、健康保健产品以及健身和户外活动的关注上。社交媒体的普及加强了社交消费的趋势。消费者更倾向于通过社交媒体平台分享购物体验、获取产品信息,这对品牌营销和产品推广产生了深远影响。随着人口老龄化,老年人口的消费需求逐渐受到关注。对于老年保健、康养服务、智能化老年产品的需求逐渐增加。

    \textbf{子问题4:城镇人口相对于农村人口的比例如何}

    如何体现:视讯会议软件的⽤户群体城镇⼈⼝相对于农村⼈⼝的⽐例?

    回答:很明显,互联⽹的发达程度决定了我们产品在该区域的使⽤程度,所以,城市的使⽤⼈⼝必然⼤于乡村,这是必然的。但是随着全⾯⼩康,像是新闻19就提到,城乡互联⽹使⽤者的社会资本的差异将会⽇渐缩⼩。农村是⼀块很⼤的潜⼒⽤户群体,在这⾥推⾏业务会⽐城市更加容易,农村也是⼀个致⼒于全⽅位功能的视讯会议软件发展⽅向。

    \textit{调研新闻 19——社交短视频的城乡使⽤者在使⽤需求以及所获得的社会资本量上存在差异,但随着短视频社交平台使⽤的不断普及和深化,社交短视频的城乡使⽤者社会资本的差异也或将逐渐缩⼩。}

    \textit{\rightline{——《中国社交短视频的使⽤对线上社会资本积累的城乡差异研究》}}

    \subsection{宏观经济影响}
    \subsubsection{全球市场影响}
    \textbf{子问题1:经济处于爆发期吗?}

    回答:在全球经济经历了激进加息、经济增速放缓、地缘冲突、石油减产等种种事件之后,黎明前的黑暗已经过去。全球经济今年的表现已经超出了他们的乐观预期,2024年增长将保持良好,通胀将继续下降,而全球主要央行们加息周期已经见顶。

    \textbf{子问题2:描述总体市场情绪}

    回答:随着全球经济逐渐从新冠疫情的冲击中恢复,一些市场表现良好,股市创下历史新高。然而,仍然存在一些令人担忧的因素,包括疫情的不确定性、通货膨胀压力、供应链问题以及地缘政治紧张局势。一些市场可能受到货币政策变化的影响,一些国家央行可能调整利率政策以适应不同的经济环境。此外,可再生能源、科技和生物技术等领域的股票可能受到投资者的关注。

    \textbf{子问题3:GDP增长率是多少}

    回答:初步核算,全年国内生产总值1210207亿元,按不变价格计算,比上年增长3.0\%。分产业看,第一产业增加值88345亿元,比上年增长4.1\%;第二产业增加值483164亿元,增长3.8\%;第三产业增加值638698亿元,增长2.3\%。分季度看,一季度国内生产总值同比增长4.8\%,二季度增长0.4\%,三季度增长3.9\%,四季度增长2.9\%

    \textbf{子问题4:失业率有多高?}

    回答:全国城镇调查失业率为5.2\%

    \subsubsection{资本市场}
    \textbf{子问题1:资本市场处于什么状态}

    如何体现:国内的视频会议的资本市场处于什么状态?

    回答:随着现在通信技术的不断进步,⼈们进⾏⽹上浏览的速度越来越快,同时⼤量的资本进⼊科技研发市场,更新技术产⽣的设备让⼈们能够对更加⾼清的画⾯进⾏欣赏。所以⼈们对视频会议的各种⽅⾯的要求会逐渐变⾼。国内的科技企业可对视频会议⾏业都开始进⾏投资,所以视频软件的资本市场⽬前是⽐较好的。
    
    \textbf{子问题2:在你所处的市场中,获得投资有多容易?}

    回答:视频会议软件符合现在的市场需求的主流,所以很多企业都希望对合理以及优质的视频会议软件⼀定的融资,通过融资能快速实现资产的翻倍,很多的企业都认为云办公在未来的市场中的占⽐⼀定会持续扩⼤。我们的软件在这个时间点进⼊市场由于市场并未完全饱和同时⼜是⽬前拥有较好发展趋势的⽅向,再融资⽅⾯不会出现太⼤的问题,只要保证⾃⼰的软件质量以及服务体验,让⽤户产⽣⼀定的粘性,能够发展出⼀定的⽤户群体,能够较为⽅便的吸引到投资⽅进⾏融资。

    \textbf{子问题3:现在就能获得种子资本、创业资本、众筹、市场资本或者贷款吗?获取这些投资的成本有多高?}

    回答:国家对于中⼩企业的贷款给予了较⼤的⽀持。因此,对于⼀个视频会议平台,这⼤⼤降低了银⾏贷款的成本值,⽽众筹所需的成本更⼩。综合⽽⾔,能够获取的⻛险投资渠道多,创业资本量⼤,获取投资的成本较低,但存在⼀定的风险。

    \textit{调研新闻——加⼤⼩微企业信贷⽀持⼒度,扩⼤续贷规模银⾏拨备和资本充⾜,不良贷款有所上升,风险可控中⼩银⾏公司治理取得初步成效,⾼⻛险机构违规股东股权清理整治稳妥开展}

    \textit{\rightline{—《国新办新闻发布会》}}

    \subsubsection{大宗商品和其他资源}
    \textbf{子问题1:描述你的业务必备的大宗商品和其他资源的当前市场状态}

    回答:视频会议所必备的资源主要是关键的技术⼈员,技术⼈员可以提供安全快速的服务同时能够保证服务的可靠性以及界⾯的美观性。现在的互联⽹企业中的劳动⼒成本在整个企业⽀出中占了⾮常⼤的⼀部分,还可以看出现在的⼈⼒成本正在持续上升,在这种情况下我们的软件和其他的软件之间的⼈才会出现争夺问题,导致市场的⼈才成本出现较为不稳定的波动。

    \textit{调研新闻 29——半数互联⽹企业的薪酬成本在企业总成本中的占⽐为 50\%~70\%,超过两成的企业薪酬成本占⽐超过 70\% 。对创业型 ⼩微企业⽽⾔,⼈⼒成本占⽐⾼的现象尤为突出:规模⼩于 30 ⼈的公司,⼈⼒成本占企业总成本⽐重往往⾼达 80\%以上。}

    \textit{\rightline{——每经⽹}}

    \textbf{子问题2:执行你的商业模式所需的资源有多容易获取?成本如何?价格走向如何?}

    回答:在⼈才⽅⾯,由于基本上所有的求职者都希望进⼊互联⽹⼤⼚,如果在这个时间进⼊市场吸引⼈才的能⼒相对较弱,所以很难对⾼精⼈才形成有效的吸引⼒。同时中国的⽼龄化社会结构⽇益严重,我们对⼈才获取的需求是⼀个⽐较重要的问题,可能会⾯临⽐较紧张的⼈才市场,同时由于刚进⼊市场时资本的不⾜,对⽐竞争对⼿出现劣势。

    \subsubsection{基础经济设施}
    \textbf{子问题1:你所处的市场基础设施有多优良}

    回答:国内的互联⽹发展速度⾮常快,我们的视频会议软件正是依托于互联⽹技术的产品,所以我们的产有者相对良好的发展环境,同时⾼速⾰新得技术也可以对我们的产品产⽣推动得效果。

    \textbf{子问题2:如何评价交通、贸易、学校质量,以及连接供应商和客户的便利度?个人和企业的税费有多高?}

    回答:和世界其他发达国家相⽐,我国⽣产税的⽐例是⽐较⾼的。我国税收侧重于流转税⽽在所得税上的弱化是企业⽣产税负较重的原因之⼀,因此也导致互联⽹企业的资⾦负担加重,⽽对个⼈税费影响较⼩。

    \section{商业模式总体评估}
    \subsection{加分项}
    客户细分:受众⼴
    
    客户关系:关系质量过硬;规模经济
    
    价值主张:产品⼈性化;产品致⼒于提⾼使⽤效率
    
    重要合作:合作范围⼴
    
    关键业务:业务贴近实际⽣活
    
    成本结构:成本相对可控
    
    收⼊来源:可期利润来源⼴
    
    渠道通路:渠道丰富
    
    \subsection{减分项}
    价值主张:严格的产品⾃我要求带来的经济和开发压⼒较⼤
    
    重要合作:冷启动需要付出较⼤的努⼒
    
    关键业务:缺少对会议环境的监督
    
    成本结构:成本效率较低
    
    核⼼资源:资源的需求较难预测

    \subsection{核心创意}

    价值主张: 腾讯会议它需要提供独特的价值,使得用户选择使用腾讯会议而不是其他竞争对手的会议软件。这可能包括优秀的视频质量、高效的云端服务、安全性能、用户友好的界面等。
    
    客户界面: 腾讯会议在客户界面方面有创意的核心,因为用户体验对于视频会议软件尤为关键。创新的用户界面、交互设计、以及方便的使用流程可能是吸引用户的关键因素。
    
    

    \section{SWOT分析}
    

    \section{蓝海战略}
    

    \section{更新过的商业模式画布}
   

    
\end{document}
